\documentclass[a4paper,12pt,titlepage]{report}
\usepackage{amsmath}
\renewcommand{\eqref}[1]{Eq.~\ref{#1}}

\begin{document}

Need to prove the following for $m < n$,

\begin{equation}
\sum_{k=0}^{n}\binom {n}{k}\left(-1\right)^{k}k^{m} =0
\end{equation}

Begin from 

\begin{equation}
\left(p+q\right)^{n}=\sum_{k=0}^{n}\binom {n}{k}p^{k}q^{n-k}
\label{binom_eq}
\end{equation}

Differentiate both sides with respect to $p$,

\begin{equation}
p\frac{d}{dp}\left(p+q\right)^{n}=pn\left(p+q\right)^{n-1}=\sum_{k=0}^{n}\binom {n}{k}kp^{k}q^{n-k}
\end{equation}

Now we let $p=-1$ and $q=-1$ to give,

\begin{equation}
\sum_{k=0}^{n}\binom {n}{k}k\left(-1\right)^{k}=0
\end{equation}

which corresponds to $m=1$ and only holds true when $n > 1$. For the general case $m$, from \eqref{binom_eq} we apply a nested derivative such that

\begin{equation}
\left(p\frac{d}{dp}\right)^{m}\left(p+q\right)^{n}=p\frac{d}{dp}\left(p\frac{d}{dp}\left(...\left(p\frac{d}{dp}\left(p+q\right)^{n} \right)\right)\right)=\sum_{k=0}^{n}\binom {n}{k}k^{m}\left(-1\right)^{k}
\end{equation}

for m=2, the LHS becomes

\begin{equation}
p\frac{d^{2}}{dp^2}\left(p+q\right)^{n}=pn\left(p+q\right)^{n-1} + p^{2}n\left(n-1\right)\left(p+q\right)^{n-2}
\end{equation}

Generally, all terms contain some form of the differential where

\begin{equation}
\frac{d^{m}}{dp^{m}}\left(p+q\right)^{n} = n\left(n-1\right)...\left(n-m+1\right)\left(p+q\right)^{n-m}
\end{equation}

Since $p=-1$ and $q=1$, for $m < n$,

\begin{equation}
\sum_{k=0}^{n}\binom {n}{k}k^{m}\left(-1\right)^{k}=0
\end{equation}


\end{document}
