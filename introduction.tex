\chapter{Introduction}

Modern experimental techniques have given us the tools to probe the microscopic scale of biomolecules. The most famous example was the use of X-ray diffraction to deduce the structure of DNA in 1953 \cite{Watson1953,Kittel1968}. Since then many mechanical experiments performed on single biomolecules have provided insights into their static structure and function \cite{Makarov2002,Cheung2004,Gaub2000}. However, within the natural system biomolecules respond dynamically to their environment to operate the mechanisms that help support biological processes, and therefore to limit discussions to zero temperature structural states is an oversimplification in understanding of the biopolymer function. By studying the free energy we are able to move towards a view that encompasses biopolymers as a dynamical system with structural disorder \cite{Beveridge1989,Kollman1993}. 

In thermodynamics the free energy is a property of a system that measures the amount of energy that can be converted into external work during an isothermal process or the system's ability to do isothermal work. The mechanism of protein folding from linear chains to unique structures that regulate cellular function can be successfully understood by the theoretical framework of the free energy landscape \cite{Berg2010,Cheung2004}. By understanding the free energy behaviour we can begin to understand the entropic effects in protein stability \cite{Karplus1987,Prevost1991,Lee2001,Trebbi2005}, protein-ligand binding \cite{Stone2001,Carlsson2005,Homans2005} and protein folding \cite{Daura2005,Bicout2000,Chavez2004}. Free energy behaviour also gives a better understanding of chemical processes that take place in drug partitioning across cell membranes. These processes are of paramount importance in the field of computer-aided drug design, and cannot be predicted reliably without knowledge of the associated free energy changes \cite{Chipot2006,Chipot2007}. Another example involves DNA. Supercoiled DNA contains a large amount of free energy that can be used to drive biological reactions \cite{Sinden1994}. The biological process of transcription and replication requires an input of work and hence a change of free energy to open and unwind the DNA double helix to expose the chemical identity of the bases at the centre of the helix \cite{Williams2010}.

While molecular simulation provides an alternative approach to describe biomolecular mechanics from a bottom-up perspective this does not easily provide a thermal description of the behaviour. We can instead use semi-analytical statistical mechanics to provide a thermodynamic model to describe biopolymer mechanics. The canonical partition function
%
\begin{equation}
\label{partition_function}
Z=\int e^{-\beta H}\, d\tau
\end{equation}
%
where we integrate over the Boltzmann factors with $H$ being the classical Hamiltonian, describes the statistical properties of a system in thermodynamic equilibrium which can be applied to biopolymers to calculate free energies for a given microstate.  It is through this method we can investigate the structural properties at finite $T$, as well as potential phase transitions brought about during DNA shearing. In this thesis we simplify the structure of DNA allowing us explicitly to write a Hamiltonian describing the partition function of the system. We focus on characterising the free energy as a function of an axial displacement $u$, thus reproducing the effect of pulling one of the two polynucleotide strands in DNA.  

Collagen is another multi-stranded biomolecule where in recent years much work has been focused on characterising the mechanical properties of collagen molecules. The structure is made up of three backbone strands twisted together into a right-handed coil known as a triple helix. Some of the earliest experiments by Sasaki and Odajima \cite{Sasaki1996} estimated the Young's modulus of collagen molecules by X-ray diffraction techniques. Others include that of Bozec and Horton \cite{Bozec2005a,Wenger2007} who used an AFM to investigate the topographical and mechanical properties of collagen molecules. However, the limitation of experimental analysis still doesn't allow the details at the nanoscale to be probed. We consider a simple structure for collagen, in which the additional strand adds an extra degree of freedom to the Hamiltonian in the partition function, increasing the complexity of the calculation. Despite this, we are still able to proceed with the calculation of the free energy, to study the axial pulling of one of the three strands in collagen. 

We begin in chapter 2 by briefly introducing ourselves to models that describe simple linear polymers. Then in chapter 3 we briefly review the Transfer Matrix Method using the Ising Model before moving on to the structure of DNA and collagen in chapter 4. In this chapter we also consider some models that describe their conformations and force-extension behaviour. Since the early 1950's numerous models have been proposed to describe linear polymer conformation and statistics, the simplest being the Freely Jointed Chain (FJC). Because the method of evaluating a partition function with a constrained mechanical extension is central to our analysis, we begin by describing our own version of the FJC using this method in chapter 5. We then provide an alternative model to the FJC, the Extendable Freely Jointed Chain (EFJC), which allows for extendable links in the chain. Chapter 6 introduces our analysis of DNA, the first of the two multi-stranded polymers investigated in this thesis. It covers the creation of a thermodynamic model for DNA extension, and describes the transfer integral method which we use to evaluate the partition function. The simplification of the DNA structure allows us to take a deeper look into the effects of base-pair breakages when a strand in the DNA molecule is pulled in an axial direction. Chapters 7 and 8 cover the analysis of collagen. We introduce a triangular prism geometric structure as a simplification of the collagen molecule and apply a similar method to calculate the partition function. We then determine the force-extension curves for a variety of inter-strand potentials. Conclusions are given in chapter 9.
