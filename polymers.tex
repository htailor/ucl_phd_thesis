\chapter{Polymers}

Studying how life on earth first originated from inert matter the Miller-Urey experiment simulated hypothetical conditions thought to be present at the time on earth over 4 billion years ago \cite{Miller1959}. Under certain conditions elements like hydrogen, carbon, oxygen and nitrogen synthesized simple organic compounds which in turn reacted to form larger structures of greater complexity. It was found that the organic compounds formed in the experiment were the same compounds that are able to go on further, and assemble more complex biomolecules such as DNA, proteins, and other biopolymers. These formed some of the earliest natural polymers in the world today. Some of these natural polymers include, nucleic acids and proteins which carry and manipulate biological information whereas polysaccharides, a polymer made of glucose as the monomeric unit, provides fuel for cell activity and serves as the structural element to living systems \cite{karp2005}. Other examples of natural polymers include starch, ligin, cellulose, collagen, silk, natural rubber, linen, and a rather more complicated example, wood \cite{Staudinger1953}.  

The discovery of synthetic polymers came in the early nineteenth century but it wouldn't be until the late 1930's that the manufacturing and use of such materials became important. Wartime demands and shortages during World War II encouraged scientists to create replacements for natural materials which were in short supply or unavailable. In this period the use of nylon, acrylic, neoprene, polyethylene, and many more polymers took the place of natural materials that were temporarily unavailable. Since then, the polymer industry has evolved into one of the fastest growing industries in the world. As research in polymeric materials continued the development of synthetic polymers accelerated to play an essential and ubiquitous role in everyday life.

Today polymers still continue to make a profound impact on industry, manufacturing, and technology.  In the field of nanotechnology, polymers are used in nano-electronics where the critical dimension scale for modern devices is now below 100 nm. Other areas include polymer-based biomaterials, nano-particle drug delivery, polymer blends and nano-composites. Nanotechnology is not new to polymer science as prior studies before the nanotechnology revolution involved nano-scale dimensions \cite{Samad2009,Paul2008}.  

\section{Polymer Architecture}

A polymer is a molecule made up by the repetition of a simpler chemical unit called a monomer \cite{bower2002}. They form to become large-chain molecules of very high molecular weight so the term "macromolecule" is often used as a synonym for polymers. If the chain molecule is made up of one type of monomer then the polymer is called a homo-polymer. If there is more than one type of monomer then it is called a copolymer \cite{Pure1996}. 

Polymers can exist in different types of chain configurations which can be either linear, branched or cross-linked. A linear polymer is a polymer molecule in which the atoms are arranged in a long chain like structure. This chain is often called the backbone. Polyethylene is a prime example of a linear polymer which is made up of ethylene monomers joined together to form the polymer. The number of monomer units in polyethylene, $n$, is usually of order of $10^{4}$, but can range between $10^{3}-10^{6}$. In some linear polymers the atoms in the chain will have small chains of atoms attached to them, much smaller than backbone. These small chains are called pendant groups which only have a few atoms. An example of a linear polymer with a pendant group is polyproylene, with it's chemical repeating unit made up of two carbon atoms, with one atom attached to two hydrogen atoms and the other carbon attached to one hydrogen, and one pendent methyl group.

\begin{figure}[htp]
\centering \includegraphics[scale=0.1]{Graphics/polyethylene-repeat-2D.png}
\caption{The stereochemistry of the ethylene monomer that makes polyethylene.}
\label{fig:polyethylene2D}
\end{figure}

\begin{figure}[htp]
\centering \includegraphics[scale=0.3]{Graphics/Polyethylene3D.png}
\caption{A 3D illustration of a polyethylene polymer.}
\label{fig:Polyethylene3D}
\end{figure}

\begin{figure}[htp]
\centering \includegraphics[scale=0.3]{Graphics/Polypropylene3D.png}
\caption{A 3D illustration of a polypropylene polymer.}
\label{fig:Polypropylene3D}
\end{figure}

A linear polymer chain may also contain some side growth that takes place from the main chain so while most of the monomeric units are linked with two others on either side, some monomeric units are linked with a third monomeric unit to form a structure that is called a branched polymer chain. These types of polymers are overwhelmingly linear but with branches attached at random points. Branched polymer molecules cannot pack together as closely as linear molecules so the intermolecular forces holding these polymers together tend to be much weaker.

Branched polymers can further change the architecture of a polymer by allowing the neighbouring polymer strands to interact. This causes the polymer strands to form chemical bonds with other other polymer strands creating a cross-linked network of chain segments in three dimensions. 

\subsection{Chemical and Geometrical Structure}

The chemical structure of polymers depends on the chemical nature of its monomeric units while the geometrical structure depends on the spatial arrangements of the adjacent monomeric units. It is quite possible to have a polymer made up of one type of monomer unit, but to have different geometrical structures. In discussing the spatial arrangement of monomeric units in a polymer chain we often talk of two important terminologies, specifically its configuration and conformation.

\section{Polymer Configuration and Conformation}
%\subsection{Polymer Configuration}

The configuration of the polymer is the arrangement fixed by chemical bonding between the adjacent monomeric units and between the atoms of the individual monomeric units. As long the chemical bonds are not broken or re-formed the configuration of the polymer remains the same. A polymer cannot shift from one configuration to another without breaking or re-forming the chemical bonds \cite{poly_sci_eng,bower2002}.  

%\subsection{Polymer Conformations}

The conformation describes the spatial arrangements of the various atoms and atomic groups in the molecule that may arise from  rotation around single bonds. In macromolecular science such conformations are called micro-conformations or local conformations since they only consist of small molecules controlled by intermolecular forces \cite{Flory1980,Elias1997}. The macro-conformation is the overall shape of the polymer molecule determined by the sum of all the micro-conformations taken by each of the repeating monomeric units.  

An interpretation of the relationships between structure and properties requires comprehensive investigation into the spatial arrangement of the atoms that make the chain molecule and its pendent groups. This essential first stage probes the connection between molecular structure and properties, not being limited to only polymers but also low molecular substances as well. Structural parameters consisting of bond lengths and bond angles may suffice to specify the geometric configuration of a molecule, but for chain molecules, especially in polymers, torsional rotations about skeletal bonds must be taken into account. The spatial configuration of a polymer depends on these angles determining its macro-conformation \cite{Flory1980}.  

At the molecular scale the configuration of a chain molecule in three dimensional space is determined by the length of the bonds $l$ in the chain backbone, by the angle specifying the difference between the directions of successive bonds $\theta$, and by the angle of rotation around the bonds $\phi$. The bond lengths $l$ and bond angles $\theta$ are fixed within narrow limits and so we can treat these quantities as pre-determined geometrical parameters. The spatial configuration of the polymer chain can therefore be determined by the fixed set of parameters $\{l,\theta\}$ and by a variable set of rotational angles $\phi_{1}...\phi_{2}...\phi_{n-2}$ about each of the $n-2$ internal bonds, $n$ being the total number of skeletal bonds in the chain.

Rotations around the single bonds $\{\phi\}$ do not occur freely, but are restricted by a rotational energy barrier determined by the bond itself and by hindrances imposed by steric interactions between local atoms and molecular groups \cite{Plischke2006}. The potentials affecting the torsional rotation of the chain's backbone also has several minima separated by barriers at least several times the thermal energy $kT$ \cite{Flory1980}. The carbon-carbon bond energy as a function of rotational angle for polyethylene is plotted in \figref{pe_potential_ethane}. 

\begin{figure}[htp]
\centering \includegraphics[scale=0.3]{Graphics/pe_potential.png}
\caption{The potential as a function of rotational angle $\phi$ around the $C-C$ bond in the backbone of polyethylene \cite{Hiemenz1984}.}
\label{fig:pe_potential_ethane}
\end{figure}

The bond-orientation at the absolute minimum of this energy where $\phi=0$, is referred to as the trans-configuration. Geometrically it is where all the three single carbon bonds $C_{n-2}-C_{n-1}-C_{n}-C_{n+1}$ lie in a plane arranged such that the carbon atoms are staggered (\figref{Polyethylene3D}). The other configuration at $\phi= \pm{2\pi/3}$ is called gauche-configurations representing a geometric shape where the bonds $C_{n-2}-C_{n-1}$ eclipse $C_{n}-C_{n+1}$. 

The rotational potential energies in polyethylene gives us two limiting cases when looking at thermal energies $RT$. In the first case where $RT$ is greater than $~16kJ/mol$ the $C-C$ bonds overcome all potential barriers allowing all the bonds in the polymer to rotate freely. Where $RT$ is less than $~4kJ/mol$ (gauche-state) the bonds occupy the lowest energy state in the trans-configuration and only move about the equilibrium position. However, the energy difference between the $trans$ and $gauche$ states give thermal energies approximately to, $RT = 2.48 kJ/mol$ at $25^{\circ}C$, which implies that at room temperature the $C-C$ bonds will be interchanging between configurations.

Polymers typically have a very large number of single bonds around which various conformations are possible, and a polymer can therefore have a very large number of conformational states. For a typical polyethylene macromolecule with an $n$ number of $C$ atoms ranging from 1000 to 100,000, the number of molecular conformations becomes extremely large, increasing as an exponential function of $n$. The number of conformational states can potentially be reduced by the steric hindrance and overlapping of the chain itself, but nevertheless, since $n$ is so large reasonably precise statements can be made about the relative probability of particular conformations of a polymer chain using statistical methods \cite{bower2002}.
