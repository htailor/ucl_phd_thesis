\chapter{Conclusion}

It has been shown in this thesis that by simplifying the structure of a multi-stranded biomolecule we are able to find an expression for the partition function needed to evaluate the free energies as a function of axial extension. The mathematical framework of using the transfer integral method to study shearing of different biomolecules has been central to our investigations. The framework was first used to model denaturation of DNA in the early 90's, and we have been able to extend the method to investigate the axial shearing of DNA and progress to study a triple stranded biomolecule. Though we have looked at two different multi-stranded biomolecules the generalisation to more strands is possible though numerically costly.

As an introduction to the partition function calculations we began with a toy model description for a Extendable Freely Jointed Chain as an improvement to the non-extendable Freely Jointed Chain. Using two top-hat weighting functions in the partition function we allowed each link in the chain to extend and contract independently. 

In chapter 6 we began with applying the transfer integral method to the DNA molecule. By simplifying the geometrical structure we were able to express the exponentiated Hamiltonian as a product of transfer matrices. By applying relevant contractions we were able to find expressions for the partition function as a function of molecular extension for cases where the DNA structure is in an intact, frayed and bubbled state. Mean base pair axial displacement calculations for the intact state gave a quantitative insight into the mechanical behaviour of the molecule under axial shear stress.  

For DNA we used two distinct methods to compare shear force data with experimental results. In the first method we used intact state calculations to determine the force when the average separation of the two end base-pairs reached a shearing length determined by $\eta_{B}$. Determining the relationship between the molecular length and the shear force we found that the results agreed well with the  experimental measurements made by Hatch et al. \cite{Hatch2008}. The values for the stiffness ratio between the backbone and base-pair that we used to fit our model to experimental data was nearly double to what was measured in \cite{Hatch2008}, but agreed well with the stiffness constant ratio fitted in the de Gennes model. 

In the second method we used the first phase transition between an intact and frayed state to determine the shear force for failure. These results produced good agreements with experimental shear force data for large molecular lengths. The stiffness ratio that we used to fit our model with the experimental data was also close to the stiffness ratio measured from experiment. However, the lack of agreement for smaller molecular lengths can be attributed to the instability of base-pairs for short DNA molecules causing large errors in the shearing force from experiment.

In Chapters 7 \& 8 we introduce the transfer integral method to the collagen molecule. By simplifying its triple helix structure to a triangular prism we study the mechanical behaviour of residue pairs under a shearing force arising from axial extension. Only studying the intact state of the molecule we formulated a general expression of the partition function using numerical hypermatrix calculations valid for all residue pair potentials as well as a harmonic approximation expression. Using data from MD simulations of collagen molecules we produced results showing the relationship between the shear force for failure and molecular length of collagen. Although no axial experiments on tropocollagen have been attempted so far the shear forces are close in magnitude to the properties other biomolecules and we gain insight into the modes of distortion of the structure. Further experimental work would be needed to test these results.

An improvement to the one dimensional models of DNA and collagen could be made by including an additional dimension that accounts for the helical structure around the axis of the biomolecule. In this thesis it was a major macroscopic feature that was largely ignored to simplify the free energy calculations, but is essentially needed to include the effects of torsion, and helical twist as we apply an extension to the structure.

The complexity of biological systems make it extremely difficult to have a single model to determine all of its behaviours and physical properties. In reality different models study different properties of the system, usually dependent on the property that is being measured, and therefore adds to the greater understanding of the system being studied.

\begin{comment}
The focus on free energy calculations in this thesis has been of significant importance in understanding biological processes when modelling biomolecules. The free energy and its derivatives are able to quantify the thermodyamic properties of a system representing the likelihood of a certain system configuration or chemical reaction. These properties are necessary to compare theory with experiment. This is an extremely difficult problem for molecular dynamics since the time-scales required for simulation are far too long to encompass all parts of configurational phase-space. Instead, the calculation of entropy change is more accurately made when have a Hamiltonian that completely describes the thermodynamic system where we can evaluate the partition function over all configurational phase-space.
\end{comment}
